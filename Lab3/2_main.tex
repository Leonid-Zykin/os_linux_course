\chapter{ОБРАБОТКА ДАННЫХ, ФАЙЛЫ И СКРИПТЫ В LINUX}

\section{Цель работы}
Кратко познакомиться с обработкой файлов и данных в Linux: чтение и фильтрация текстов командами, базовые конструкции Python-скриптов из раздатки, а также практическая работа с файлами CSV.

\section{Ход выполнения}

\subsection{Подготовка окружения}
- Открыт терминал и переход в каталог лабораторной работы: \texttt{cd Lab3}
- Проверены входные данные: \texttt{ls -l}, наличие файлов \texttt{lab3.py}, \texttt{lab3\_2.py}, \texttt{the-matrix-reloaded.csv}

\subsection{Работа с текстовыми утилитами}
- Просмотр первых/последних строк файла: \texttt{head}, \texttt{tail}
- Подсчет строк и столбцов: \texttt{wc -l}, \texttt{cut -d, -fN}
- Поиск по шаблону: \texttt{grep 'pattern' file}
- Сортировка и уникализация: \texttt{sort | uniq -c}

\subsection{Скрипты Python}
- Запущен основной скрипт: \texttt{python3 lab3.py}, изучен вывод и структура кода
- Дополнительно выполнен \texttt{python3 lab3\_2.py} для обработки CSV (агрегации/фильтры)
- Проверены аргументы запуска (при наличии): \texttt{python3 lab3.py --help}

\subsection{Обработка CSV}
- Просмотр заголовков: \texttt{head -n 1 the-matrix-reloaded.csv}
- Извлечение нужных столбцов: \texttt{cut -d, -f1,3 the-matrix-reloaded.csv}
- Подсчет количества записей по условию с \texttt{grep} и \texttt{wc -l}
- Сохранение результатов в файл отчета: \texttt{> results.txt}

\section{Ответы и пояснения}
- Использовали стандартные утилиты \texttt{head/tail/grep/cut/sort/uniq} для базовой аналитики
- Скрипты на Python позволили автоматизировать операции над данными и проверить корректность ручных шагов
- Файл CSV обрабатывался как потоки: выделение столбцов разделителем, фильтрация по регулярным выражениям, агрегирование

\section{Вывод}
Проделана небольшая практика по обработке данных в Linux с помощью утилит командной строки и Python. Получены навыки фильтрации, разбиения на столбцы, подсчета и агрегации, а также запуска и проверки скриптов.

\section{Полученные файлы и результаты (фрагменты/полный вывод)}

\subsection{Вывод стандартного потока (zykin\_out)}
\lstinputlisting[language={},basicstyle=\ttfamily\small]{work/zykin_out}

\subsection{Вывод стандартного потока ошибок (zykin\_err)}
\lstinputlisting[language={},basicstyle=\ttfamily\small]{work/zykin_err}

\subsection{Итоговый файл (zykin\_final)}
\lstinputlisting[language={},basicstyle=\ttfamily\small]{work/zykin_final}

\subsection{Текстовая выборка (keanu\_text, первые 20 строк)}
\lstinputlisting[firstline=1,lastline=20,language={},basicstyle=\ttfamily\small]{work/keanu_text}

\subsection{CSV-данные (the-matrix-reloaded.csv, первые 20 строк)}
\lstinputlisting[firstline=1,lastline=20,language={},basicstyle=\ttfamily\small]{work/the-matrix-reloaded.csv}

\subsection{Скрипт lab3.py}
\lstinputlisting[language=Python]{work/lab3.py}

\subsection{Скрипт lab3\_2.py}
\lstinputlisting[language=Python]{work/lab3_2.py}
