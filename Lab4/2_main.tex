\chapter{ОСНОВЫ БЕЗОПАСНОСТИ В LINUX}

\section{Цель работы}
Изучить управление пользователями и группами, права доступа к файлам, а также базовые операции с процессами. Зафиксировать ход работы в файлах записи терминала.

\section{Краткий ход выполнения}
\subsection{Права root и системные файлы}
- Просмотр \texttt{/etc/passwd} с пояснением полей: \texttt{имя:х:UID:GID:комментарий:домашний\_каталог:оболочка}
- Использование \texttt{sudo} для безопасного выполнения привилегированных команд

\subsection{Пользователи и группы}
- Созданы пользователи: \texttt{user1}, \texttt{user2}, \texttt{user3}
- Создана группа \texttt{Students}, добавлены \texttt{user1}, \texttt{user2}
- \texttt{user3} получал root-права для административных шагов
- Проверена принадлежность к группам командой \texttt{groups}

\subsection{Файлы и права}
- Под \texttt{user1} создан файл \texttt{schedule}, владельцем назначена группа \texttt{Students} (\texttt{chgrp})
- Под \texttt{user2} проверена доступность и выполнена попытка записи
- Под \texttt{user3} создан файл \texttt{important}, права изменены \texttt{chmod} для доступа Others
- Исследованы умолчания прав и отличия прав у файлов, созданных с root

\subsection{Процессы}
- Получен список процессов и сохранён в \texttt{proc\_list}
- Изучены \texttt{pstree}, \texttt{kill}, \texttt{killall}



\subsection{Учебные файлы}
\lstinputlisting[language={},basicstyle=\ttfamily\small]{schedule}
\lstinputlisting[language={},basicstyle=\ttfamily\small]{important}

\subsection{Список процессов}
\lstinputlisting[language={},basicstyle=\ttfamily\small]{proc_list}

\section{Полученные файлы (фрагменты/полный вывод)}
\subsection{Файлы записи терминала}
\lstinputlisting[language={},basicstyle=\ttfamily\small]{l4_result_u1_clean_2}
\lstinputlisting[language={},basicstyle=\ttfamily\small]{l4_result_u2_clean_2}
\lstinputlisting[language={},basicstyle=\ttfamily\small]{l4_result_u3_clean_2}

\section{Вывод}
Выполнены основные действия по управлению пользователями/группами, настройке прав доступа к файлам и работе с процессами. Результаты шагов зафиксированы в приложенных файлах записи.
